\documentclass[a4paper, 12pt]{extarticle}
  \usepackage{datetime}
  \usepackage[utf8]{inputenc}
  \usepackage[T1]{fontenc}
  \usepackage{lmodern}
  \title{Texte argumentatif: la consommation}
  \author{Esteban Sotillo \\
    CPNV - Français 
    }
  
  \date{\today}
  % Hint: \title{what ever}, \author{who care} and \date{when ever} could stand 
  % before or after the \begin{document} command 
  % BUT the \maketitle command MUST come AFTER the \begin{document} command! 


\begin{document}
\maketitle
\begin{abstract}\begin{center}\Large{}
	Dans quelle mesure la consommation d'alcool est-elle néfaste pour notre société ? 
	\Large{}\end{center}\end{abstract}
			
		  
	\section{Introduction}
	Dans notre société actuelle, plus de 60\% de la population consomme régulièrement (au moins une fois par semaine) de l'alcool.\\
	Sachant que l'alcool a des effets néfaste sur le corps et l'esprit en grande quantité mais bénéfique quand il est modéré, il est intéressant de regarder plus en profondeur les effets produits sur notre société.\\
	Nous allons donc nous pencher sur les dégâts que peut provoquer la consommation d'alcool sur la santé à court et long terme.\\
	Puis, nous nous intéresserons aux effets d'une plus ou moins forte consommation dans le domaine public, professionnel mais aussi privé (en ménage).\\
	Pour terminer, nous jetterons un \oe il sur l'évolution des taxes de ce produit pour en déterminer l'utilité dans la société. 
  \section{La santé}
  Pour commencer, nous allons voir les effets de l'alcool sur le corps.
  \subsection{Court terme}
  L'alcool, par sa nature, agit sur le cerveau et altère donc ses capacités. De manière générale, il provoque différents effets selon la quantité ingurgitée. Les effets sont bien connus en général mais peuvent être classé en 3 catégories.\\
  La phase d'\em{}excitation\em{} est le premier stade, tant que le taux d'alcool ne dépasse pas 0.7 [g/l]\footnote{gramme par litre de sang}. Cet état produit de l'euphorie, de la désinhibition, de la bavardise et de la familiarité. Dans le même temps différents fonctions cognitives sont altérées : la vigilance, la perception, la mémoire et l'équilibre. Cela à pour effets de baisser la concentration et la mémoire au profit de la créativité.\\
  Ensuite, arrive le phase d'\em{}ébriété\em{} qui s'étend à un taux entre 0.7 [g/l] et 2 [g/l]. Cette phase d'incoordination provoque des troubles aigus de la vigilance, de la mémoire, et de la parole.\\
  La dernière phase est l'\em{}endormissement\em{} provoquée par un taux d'alcool supérieur à 2 [g/l] et accentue simplement les effets précédents et provoque des somnolences jusqu'à atteindre un taux de 3 [g/l] où il y a un risque de coma éthylique.\\ 
  A court terme, la consommation modérée ne semble pas mauvaise pour la société mais dès que le taux dépasse les 0.7 [g/l] tout change et la personne alcoolisée risque de devenir un danger pour autrui ou sois-même.
  \subsection{Long terme}
  Dans un deuxième temps, l'alcool agit aussi à petites doses. Il provoques d'autres dégâts sur le long terme.
  Boire régulièrement ,dès un verre d'alcool par jour, peut provoquer un cirrhose ou même un cancer. Cette consommation régulière peut aussi mener à une dépression. Une dépression qui peut, paradoxalement, entrainer une dépendance du a la volonté d' "oublier" les causes de cette dépression.\\
  La consommation d'alcool reguilère augmente aussi la pression sanguine impliquant une hausse des chances de faire un AVC.\\
  Il existe aussi un mythe populaire qui dit que un verre de vin rouge par repas baisserais le risque de problème cardiovasculaire. Ceci dit, aucune études n'a démontrer une réel existance de ce phénomène avec le vin rouge.\\
  En Bref, consommer de l'alcool réfuilièrement et nocif pour la santé personnel dans tous les cas sans compter les dégâts liés au cerveau. Tous ces effets sur la santé vont impliquer réactions différentes selon l'environnement.
  \section{Les environnements}
  Dans chaque environnement de la vie quotidienne l'alcool affecte différemment la société.
  \subsection{Public}
  Dans le domaine public ou plus simplement dans la rue, l'alcool affecte de plusieures façons notre société. Par exemple, la conduite en étât d'ébriété augmente de plus de 6 fois les risques d'accidents de voiture.\\
  En état d'ivresse, se balader dans la rue devient plus dangereux dû au fait que nous ne sommes plus asser suffisemment et réactif afin de se défendre correctement.\\
  Une fois en extérieure, la consommation excessive augmente grandement les risques de danger pour sois-même comme pour autrui.
  \subsection{Professionnel}
  L'alcool au travail est proscrite ou très limitée dans la plupart des entreprises, cependant, les employés consomment la veille au soir. Ces apéros nocturnes coutent approximativement 4.2\footnote{selon l'Office Fédéral de la Santé Publique (OFSP)} milliards de CHF à la société\footnote{Ici la Confédération Helvétique} et 3.4 milliards de CHF aux entreprises suisses. Ces sommes faramineuse comportent les frais médicaux, les accidents et blessures, les baisses de performances et les absences dûes à l'alcool.\\
  Malgré ces gros coûts, l'alcool peut être benéfique dans un seul cas pour une entreprise, en petite quantité (0.75 [g/l]) elle permet de booster la créativité au détriment de la mémoire.
  \subsection{Privé}
  Les foyers consomment régulièrement un verre d'alcool lors de repas créant  souvent une ambiance plus agréable grâce aux effets encouragent le bavardage si consommé modérément.\\
  L'alcool à la maison impacte grandement la vie familiale uniquement lors d'alcoolémisme où les chances de violences au foyer vont augmenter.\\
  En bref, la consommation d'alcool est un problème sociétaire dès que la quantité n'est pas modérée mais, à l'inverse, peut-être un atout si elle l'est. 
	\section{Les taxes}
	\subsection{Actuellement}  
	\subsection{Avant}
	\section{Conclusion}
		  
	%\subsection{b}
	%\subsubsection{c}
		
	\pagebreak
      
  \tableofcontents

	\begin{thebibliography}{9}
		\bibitem{drunknotblind} \emph{Drunk, but not blind}, \begin{scriptsize}[article, anglais]\end{scriptsize}\\
    \begin{scriptsize}\texttt{https://www.sciencedirect.com/science/article/pii/S1053810013000032 }\end{scriptsize}

    \bibitem{risquecourtterme} \emph{Risque à court terme}, \begin{scriptsize}[article]\end{scriptsize}\\
    \begin{scriptsize}\texttt{http://www.alcool-info-service.fr/alcool/consequences-alcool/risques-court-terme\#.WpWQaXWYVhE} \end{scriptsize}

    \bibitem{risquelongterme} \emph{Risque à long terme}, \begin{scriptsize}[article]\end{scriptsize}\\
    \begin{scriptsize}\texttt{http://www.alcool-info-service.fr/alcool/consequences-alcool/risques-long-terme\#.WpWSeHWYVhE} \end{scriptsize}

    \bibitem{degatttetages} \emph{Dégâts à tout les étages}, \begin{scriptsize}[article]\end{scriptsize}\\
    \begin{scriptsize}\texttt{http://www.doctissimo.fr/html/dossiers/alcool/sa\_6455\_alcool\_consequences\_sante.htm} \end{scriptsize}

    \bibitem{consosuissexls} \emph{Consommation d'alcool en suisse}, \begin{scriptsize}[xls]\end{scriptsize}\\
    \begin{scriptsize}\texttt{https://www.bfs.admin.ch/bfsstatic/dam/assets/302777/master} \end{scriptsize}

	\end{thebibliography}
\end{document}