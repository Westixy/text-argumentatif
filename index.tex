\documentclass[a4paper, 14pt]{extarticle}
  \usepackage{datetime}
  \usepackage[utf8]{inputenc}
  \usepackage[T1]{fontenc}
  \usepackage{lmodern}
  \title{Texte argumentatif: la consommation}
  \author{Esteban Sotillo \\
    CPNV - Français 
    }
  
  \date{\today}
  % Hint: \title{what ever}, \author{who care} and \date{when ever} could stand 
  % before or after the \begin{document} command 
  % BUT the \maketitle command MUST come AFTER the \begin{document} command! 


\begin{document}
\maketitle
\begin{abstract}\begin{center}
	Dans quelle mesure la consommation d'alcool est-elle néfaste pour notre société ? 
	\end{center}\end{abstract}
			
	\pagebreak
		  
	\section{Introduction}
	Dans notre société actuelle, plus de 60\% de la population consomme régulièrement (au moins une fois par semaine) de l'alcool.\\
	Sachant que l'alcool a des effets néfaste sur le corps et l'esprit en grande quantité mais bénéfique quand il est modéré, il est intéressant de regarder plus en profondeur les effets produits sur notre société.\\
	Nous allons donc nous pencher sur les dégâts que peut provoquer la consommation d'alcool sur la santé à court et long terme.\\
	Puis, nous nous intéresserons aux effets d'une plus ou moins forte consommation dans le domaine public, professionnel mais aussi privé (en ménage).\\
	Pour terminer, nous jetterons un \oe il sur l'évolution des taxes de ce produit pour en déterminer l'utilité dans la société. 
	\pagebreak
  \section{La santé}
  Pour commencer, nous allons voir les effets de l'alcool sur le corps.
  \subsection{Court terme}
  L'alcool, par sa nature, agit sur le cerveau et altère donc ses capacités. De manière générale, il provoque différents effets selon la quantité ingurgitée. Les effets sont bien connus en général mais peuvent être classé en 3 catégories.\\
  La phase d'\em{}excitation\em{} est le premier stade, tant que le taux d'alcool ne dépasse pas 0.7 [g/l]\footnote{gramme par litre de sang}. Cet état produit de l'euphorie, de la désinhibition, de la bavardise et de la familiarité. Dans le même temps différents fonctions cognitives sont altérées : la vigilance, la perception, la mémoire et l'équilibre. Cela à pour effets de baisser la concentration et la mémoire au profit de la créativité.\\
  Ensuite, arrive le phase d'\em{}ébriété\em{} qui s'étend à un taux entre 0.7 [g/l] et 2 [g/l]. Cette phase d'incoordination provoque des troubles aigus de la vigilance, de la mémoire, et de la parole.\\
  La dernière phase est l'\em{}endormissement\em{} provoquée par un taux d'alcool supérieur à 2 [g/l] et accentue simplement les effets précédents et provoque des somnolences jusqu'à atteindre un taux de 3 [g/l] où il y a un risque de coma éthylique.\\
  A court terme, la consommation modérée ne semble pas mauvaise pour la société mais dès que le taux dépasse les 0.7 [g/l] tout change et la personne alcoolisée risque de devenir un danger pour autrui ou sois-même.
	\subsection{Long terme}
	\section{Les environnements}
	\subsection{Public}
	\subsection{Professionnel}
	\subsection{Privé}
	\section{Les taxes}
	\subsection{Actuellement}  
	\subsection{Avant}
	\section{Conclusion}
		  
	%\subsection{b}
	%\subsubsection{c}
		
	\pagebreak
      
  \tableofcontents

	\begin{thebibliography}{9}
		\bibitem{drunknotblind} \emph{Drunk, but not blind}, \begin{scriptsize}[article, anglais]\end{scriptsize}\\
    \begin{scriptsize}\texttt{https://www.sciencedirect.com/science/article/pii/S1053810013000032 }\end{scriptsize}

    \bibitem{risquecourtterme} \emph{Risque à court terme}, \begin{scriptsize}[article]\end{scriptsize}\\
    \begin{scriptsize}\texttt{http://www.alcool-info-service.fr/alcool/consequences-alcool/risques-court-terme\#.WpWQaXWYVhE} \end{scriptsize}

    \bibitem{risquelongterme} \emph{Risque à long terme}, \begin{scriptsize}[article]\end{scriptsize}\\
    \begin{scriptsize}\texttt{http://www.alcool-info-service.fr/alcool/consequences-alcool/risques-long-terme\#.WpWSeHWYVhE} \end{scriptsize}

    \bibitem{degatttetages} \emph{Dégâts à tout les étages}, \begin{scriptsize}[article]\end{scriptsize}\\
    \begin{scriptsize}\texttt{http://www.doctissimo.fr/html/dossiers/alcool/sa\_6455\_alcool\_consequences\_sante.htm} \end{scriptsize}

    \bibitem{consosuissexls} \emph{Consommation d'alcool en suisse}, \begin{scriptsize}[xls]\end{scriptsize}\\
    \begin{scriptsize}\texttt{https://www.bfs.admin.ch/bfsstatic/dam/assets/302777/master} \end{scriptsize}

	\end{thebibliography}
\end{document}