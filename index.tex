\documentclass[a4paper, french, 12pt]{extarticle}
  \usepackage{datetime}
  \usepackage{babel}
  \usepackage[utf8]{inputenc}
  \usepackage[T1]{fontenc}
  \usepackage{lmodern}
  \title{Texte argumentatif: la consommation}
  \author{Esteban Sotillo \\
    CPNV - Français 
    }
  
  \date{\today}
  % Hint: \title{what ever}, \author{who care} and \date{when ever} could stand 
  % before or after the \begin{document} command 
  % BUT the \maketitle command MUST come AFTER the \begin{document} command! 


\begin{document}
\maketitle
\begin{abstract}\begin{center}\Large{}
	Dans quelle mesure la consommation d'alcool est-elle néfaste pour notre société ? 
	\Large{}\end{center}\end{abstract}
				
			  
	\section{Introduction}
	Dans notre société actuelle, plus de 60\% de la population consomme régulièrement (au moins une fois par semaine) de l'alcool.\\
	Sachant que l'alcool a des effets néfaste sur le corps et l'esprit en grande quantité mais bénéfique quand il est modéré, il est intéressant de regarder plus en profondeur les effets produits sur notre société.\\
	Nous allons donc nous pencher sur les dégâts que peut provoquer la consommation d'alcool sur la santé à court et long terme.\\
	Puis, nous nous intéresserons aux effets d'une plus ou moins forte consommation dans le domaine public, professionnel mais aussi privé (en ménage).\\
	Pour terminer, nous jetterons un \oe il sur l'évolution des taxes de ce produit pour en déterminer l'utilité dans la société. 
	\section{La santé}
	Pour commencer, nous allons voir les effets de l'alcool sur le corps.
	\subsection{Court terme}
	L'alcool, par sa nature, agit sur le cerveau et altère donc ses capacités. De manière générale, il provoque différents effets selon la quantité ingurgitée. Les effets sont bien connus en général mais peuvent être classé en 3 catégories.\\
	La phase d'\em{}excitation\em{} est le premier stade, tant que le taux d'alcool ne dépasse pas 0.7 [g/l]\footnote{gramme par litre de sang}. Cet état produit de l'euphorie, de la désinhibition, de la bavardise et de la familiarité. Dans le même temps différents fonctions cognitives sont altérées : la vigilance, la perception, la mémoire et l'équilibre. Cela à pour effets de baisser la concentration et la mémoire au profit de la créativité.\\
	Ensuite, arrive le phase d'\em{}ébriété\em{} qui s'étend à un taux entre 0.7 [g/l] et 2 [g/l]. Cette phase d'incoordination provoque des troubles aigus de la vigilance, de la mémoire, et de la parole.\\
	La dernière phase est l'\em{}endormissement\em{} provoquée par un taux d'alcool supérieur à 2 [g/l] et accentue simplement les effets précédents et provoque des somnolences jusqu'à atteindre un taux de 3 [g/l] où il y a un risque de coma éthylique.\\ 
	A court terme, la consommation modérée ne semble pas mauvaise pour la société mais dès que le taux dépasse les 0.7 [g/l] tout change et la personne alcoolisée risque de devenir un danger pour autrui ou sois-même.
	\subsection{Long terme}
	Dans un deuxième temps, l'alcool agit aussi à petites doses. Il provoques d'autres dégâts sur le long terme.
	Boire régulièrement ,dès un verre d'alcool par jour, peut provoquer un cirrhose ou même un cancer. Cette consommation régulière peut aussi mener à une dépression. Une dépression qui peut, paradoxalement, entraîner une dépendance du a la volonté d' "oublier" les causes de cette dépression.\\
	La consommation d'alcool régulière augmente aussi la pression sanguine impliquant une hausse des chances de faire un AVC.\\
	Il existe aussi un mythe populaire qui dit que un verre de vin rouge par repas baisserais le risque de problème cardiovasculaire. Ceci dit, aucune études n'a démontrer une réel existence de ce phénomène avec le vin rouge.\\
	En Bref, consommer de l'alcool régulièrement et nocif pour la santé personnel dans tous les cas sans compter les dégâts liés au cerveau. Tous ces effets sur la santé vont impliquer réactions différentes selon l'environnement.
	\section{Les environnements}
	Dans chaque environnement de la vie quotidienne l'alcool affecte différemment la société.
	\subsection{Public}
	Dans le domaine public ou plus simplement dans la rue, l'alcool affecte de plusieurs façons notre société. Par exemple, la conduite en état d'ébriété augmente de plus de 6 fois les risques d'accidents de voiture.\\
	En état d'ivresse, se balader dans la rue devient plus dangereux dû au fait que nous ne sommes plus suffisamment et réactif afin de se défendre correctement.\\
	Une fois en extérieure, la consommation excessive augmente grandement les risques de danger pour sois-même comme pour autrui.
	\subsection{Professionnel}
	L'alcool au travail est proscrite ou très limitée dans la plupart des entreprises, cependant, les employés consomment la veille au soir. Ces apéros nocturnes coûtent approximativement 4.2\footnote{selon l'Office Fédéral de la Santé Publique (OFSP)} milliards de CHF à la société\footnote{Ici la Confédération Helvétique} et 3.4 milliards de CHF aux entreprises suisses. Ces sommes faramineuse comportent les frais médicaux, les accidents et blessures, les baisses de performances et les absences dues à l'alcool.\\
	Malgré ces gros coûts, l'alcool peut être bénéfique dans un seul cas pour une entreprise, en petite quantité (0.75 [g/l]) elle permet de booster la créativité au détriment de la mémoire.
	\subsection{Privé}
	Les foyers consomment régulièrement un verre d'alcool lors de repas créant  souvent une ambiance plus agréable grâce aux effets encouragent le bavardage si consommé modérément.\\
	L'alcool à la maison impacte grandement la vie familiale uniquement lors d'alcoolémie où les chances de violences au foyer vont augmenter causer par l'altération du jugement.\\
	En bref, la consommation d'alcool est un problème sociétaire dès que la quantité n'est pas modérée mais, à l'inverse, peut-être un atout si elle l'est dans des cas particulier.
	\section{Les taxes}
	L’histoire montre qu’en Suisse, la société et la politique n’ont pas toujours considéré l’alcool comme un problème de même poids.
	\subsection{Avant}
	En 1885, une votation populaire sur l'imposition de l'alcool a été acceptée. A ce moment là, la question est discutée comme un problème social qui vise spécialement les paysans et ouvriers. Depuis lors, le peuple a voté 17 fois au niveau fédéral en environ 130 ans. C'est le produit qui a été le plus controversé de Suisse.\\
	Après la première guerre mondiale, la régie fédéral des alcools vise une politique de modération qui n'a pour seul but la réduction de la consommation d'alcool. A ce moment là, 10\% des recettes sont reversée pour la lutte contre l'alcool.\\
	Jusqu'en 1980, la fédération ne s'occupe que de l'alcool distillé. Le problème est qu'à l'origine, le vin et la bière étaient peu consommé par rapport au spiritueux. Mais en 1980, ceux-ci ne représentent plus qu'un cinquième du marché. Il fût donc nécessaire de réajuster les lois les concernant pour taxer l'intégralité de l'alcool. La taxe change aussi à 5\% du bénéfice net généré.\\
	En 2003, la taxe va changer brutalement et augmenter de 300\% pour les "alcopops".
	\subsection{Actuellement}
	Dans l'état actuelle des choses, l'alcool est imposé à 29 CHF par litre d'alcool pur pour les spiritueux et à environ 0.25 CHF par litre de bière.\\
	Selon l'article 131\up{3} de la Constitution fédéral:
	\begin{quote}\begin{em}
		Un dixième du produit net de l'impôt sur les boissons distillées est versé aux cantons. Ils utilisent ces fonds pour combattre les causes et les effets de l'abus de substances engendrant la dépendance.
  \end{em}\end{quote}
  Ce qui a pour but de compenser une parties des pertes engendrées par l'alcool pour la société.
  Il existe aussi une taxe sur l'emballage servant aux coûts d'élimination et de recyclage des verres usagés.\\
  Il est donc visible que la société a visé la modération d'alcool sans pour autant viser à supprimer cette consommation qui fais débats depuis plus d'un siècle dans notre pays. Du moins, pas en utilisant les taxes pour ce faire. Cependant, les taxes sont utiles uniquement pour amoindrir le coût des dégâts que produit le liquide sur la population.
	
  \section{Conclusion}
  Finalement, pour revenir à la question d'origine, il est claire que l'alcool est néfaste pour la société. Consommer de l'alcool est resté presque comme un tradition en Suisse tout en suivant une, faible mais constante, diminution de la quantité consommée par personne a travers les années.\\
  Nous avons vu que la société ne vise pas non plus la suppression de l'alcool mais plus sa modération. Même si une récente étude a montré que dans certains cas rare l'alcool peut être bénéfique il reste principalement une source de problème pour la société mais est apprécié par la population.\\
  Il reste a voir si la consommation d'alcool va continuer sa lente diminution ou va se stabiliser avec le temps.\\
  L'alcool sera certainement toujours présent tant que du monde sera là pour le boire.   
			  
	%\subsection{b}
	%\subsubsection{c}
			
	\pagebreak
	      
	\tableofcontents
	
	\begin{thebibliography}{9}
		\bibitem{drunknotblind} \emph{Drunk, but not blind}, \begin{scriptsize}[article, anglais]\end{scriptsize}\\
		\begin{tiny}\texttt{https://www.sciencedirect.com/science/article/pii/S1053810013000032 }\end{tiny}
		
		\bibitem{risquecourtterme} \emph{Risque à court terme}, \begin{scriptsize}[article]\end{scriptsize}\\
		\begin{tiny}\texttt{http://www.alcool-info-service.fr/alcool/consequences-alcool/risques-court-terme\#.WpWQaXWYVhE}\end{tiny}
		
		\bibitem{risquelongterme} \emph{Risque à long terme}, \begin{scriptsize}[article]\end{scriptsize}\\
		\begin{tiny}\texttt{http://www.alcool-info-service.fr/alcool/consequences-alcool/risques-long-terme\#.WpWSeHWYVhE}\end{tiny}
		
		\bibitem{degatttetages} \emph{Dégâts à tout les étages}, \begin{scriptsize}[article]\end{scriptsize}\\
		\begin{tiny}\texttt{http://www.doctissimo.fr/html/dossiers/alcool/sa\_6455\_alcool\_consequences\_sante.htm} \end{tiny}
		
		\bibitem{consosuissexls} \emph{Consommation d'alcool en suisse}, \begin{scriptsize}[xls]\end{scriptsize}\\
		\begin{tiny}\texttt{https://www.bfs.admin.ch/bfsstatic/dam/assets/302777/master} \end{tiny}
		
		\bibitem{taxessuisse} \emph{Historique impôts Fédéraux}, \begin{scriptsize}[pdf]\end{scriptsize}\\
		\begin{tiny}\texttt{https://www.estv.admin.ch/dam/estv/fr/dokumente/allgemein/Dokumentation/Publikationen/\\dossier\_steuerinformationen/a/Daten\%20aus\%20der\%20Geschichte\%20der\%20Bundessteuern.pdf\\.download.pdf/a\_daten\_geschichte\_f.pdf} \end{tiny}
		
	\end{thebibliography}
\end{document}